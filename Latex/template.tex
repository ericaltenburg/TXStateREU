\documentclass{article}


\usepackage{arxiv}

\usepackage[utf8]{inputenc} % allow utf-8 input
\usepackage[T1]{fontenc}    % use 8-bit T1 fonts
\usepackage{hyperref}       % hyperlinks
\usepackage{url}            % simple URL typesetting
\usepackage{booktabs}       % professional-quality tables
\usepackage{amsfonts}       % blackboard math symbols
\usepackage{nicefrac}       % compact symbols for 1/2, etc.
\usepackage{microtype}      % microtypography
\usepackage{lipsum}

\title{Pre-flight Battery Consumption Model for UAV Missions}


\author{
  Eric R. Altenburg\\
  Computer Science\\
  Stevens Institute of Technology\\
  Hoboken, NJ \\
  \texttt{ealtenbu@stevens.edu}\\
   \And
  John H. Graves\\
  Computer Science\\
  Brown University\\
  Providence, RI \\
  \texttt{john\textunderscore graves@brown.edu} \\
  \And
  Qijun Gu\\
  Computer Science\\
  Texas State University\\
  San Marcos, TX \\
  \texttt{qg11@txstate.edu}\\
}

\begin{document}
\maketitle

\begin{abstract}
This is the abstract...
\end{abstract}


% keywords can be removed
\keywords{First keyword \and Second keyword \and More}


\section{Introduction}
While there are methods that allow a user to observe the state of charge on a given unmanned aerial vehicle’s (UAV) battery in real-time, there is no accurate or efficient approach to predict the total battery consumption of planned UAV missions.\par

As UAV technology improves, these devices become more and more integrated into society as they allow for tasks to be easily completed by a user in a remote location. For example, Amazon and other distribution companies are developing methods that use these devices to deliver packages to their users in a fast and simple manner. It is speculated that if UAVs are to replace delivery trucks, then not only will it cut down on cost but it will have a positive effect on the environment as well due to fewer emissions being released by these vehicles compared to standard trucks (MIGHT NEED TO BACK THIS UP, I’M JUST HYPING UP THE DRONES). However, UAVs are not exclusive to corporate delivery systems, they can also be used for sport surveillance purposes and police work in situations where it might not be safe to send a human being in an area; such as a bomb threat.\par

By knowing the predicted battery consumption, the user is at a much greater advantage as they can determine whether a mission is feasible or if the UAV will require a battery change.


\section{Background}
\label{sec:headings}

Background Info \par
See Section \ref{sec:headings}.

\subsection{Previous Research}
Previous research \par
Equation Example: \newline
\begin{equation}
\xi _{ij}(t)=P(x_{t}=i,x_{t+1}=j|y,v,w;\theta)= {\frac {\alpha _{i}(t)a^{w_t}_{ij}\beta _{j}(t+1)b^{v_{t+1}}_{j}(y_{t+1})}{\sum _{i=1}^{N} \sum _{j=1}^{N} \alpha _{i}(t)a^{w_t}_{ij}\beta _{j}(t+1)b^{v_{t+1}}_{j}(y_{t+1})}}
\end{equation}


\paragraph{Paragraph}
The Mentioned paragraph continues onto muliple lines it is a paragraph after all. What else do I write? I don't know

\section{Methods}
\label{sec:others}
When you have a paragraph you can also cite it. \cite{kour2014real,kour2014fast} and see \cite{hadash2018estimate}.

And for a URL, the documentation for \verb+natbib+ may be found at
\begin{center}
  \url{http://mirrors.ctan.org/macros/latex/contrib/natbib/natnotes.pdf}
\end{center}
Of note is the command \verb+\citet+, which produces citations
appropriate for use in inline text.  For example, 
\begin{verbatim}
   \citet{hasselmo} investigated\dots
\end{verbatim}
produces
\begin{quote}
  Hasselmo, et al.\ (1995) investigated\dots
\end{quote}

\begin{center}
  \url{https://www.ctan.org/pkg/booktabs}
\end{center}


\subsection{Figures}
\lipsum[10] 
See Figure \ref{fig:fig1}. Here is how you add footnotes. \footnote{Sample of the first footnote.}
\lipsum[11] 

\begin{figure}
  \centering
  \fbox{\rule[-.5cm]{4cm}{4cm} \rule[-.5cm]{4cm}{0cm}}
  \caption{Sample figure caption.}
  \label{fig:fig1}
\end{figure}


\subsection{Tables}
\lipsum[12]
See awesome Table~\ref{tab:table}.

\begin{table}
 \caption{Sample table title}
  \centering
  \begin{tabular}{lll}
    \toprule
    \multicolumn{2}{c}{Part}                   \\
    \cmidrule(r){1-2}
    Name     & Description     & Size ($\mu$m) \\
    \midrule
    Dendrite & Input terminal  & $\sim$100     \\
    Axon     & Output terminal & $\sim$10      \\
    Soma     & Cell body       & up to $10^6$  \\
    \bottomrule
  \end{tabular}
  \label{tab:table}
\end{table}

\subsection{Lists}
\begin{itemize}
\item Lorem ipsum dolor sit amet
\item consectetur adipiscing elit. 
\item Aliquam dignissim blandit est, in dictum tortor gravida eget. In ac rutrum magna.
\end{itemize}


\bibliographystyle{unsrt}  
%\bibliography{references}  %%% Remove comment to use the external .bib file (using bibtex).
%%% and comment out the ``thebibliography'' section.


%%% Comment out this section when you \bibliography{references} is enabled.
\begin{thebibliography}{1}

\bibitem{kour2014real}
George Kour and Raid Saabne.
\newblock Real-time segmentation of on-line handwritten arabic script.
\newblock In {\em Frontiers in Handwriting Recognition (ICFHR), 2014 14th
  International Conference on}, pages 417--422. IEEE, 2014.

\bibitem{kour2014fast}
George Kour and Raid Saabne.
\newblock Fast classification of handwritten on-line arabic characters.
\newblock In {\em Soft Computing and Pattern Recognition (SoCPaR), 2014 6th
  International Conference of}, pages 312--318. IEEE, 2014.

\bibitem{hadash2018estimate}
Guy Hadash, Einat Kermany, Boaz Carmeli, Ofer Lavi, George Kour, and Alon
  Jacovi.
\newblock Estimate and replace: A novel approach to integrating deep neural
  networks with existing applications.
\newblock {\em arXiv preprint arXiv:1804.09028}, 2018.

\end{thebibliography}

%\ack{kjlfsjls}


\end{document}
